%In R, use require("knitr") and knit("this Rnw file") to generate the tex file

\documentclass[12pt]{article}\usepackage[]{graphicx}\usepackage[]{color}
%% maxwidth is the original width if it is less than linewidth
%% otherwise use linewidth (to make sure the graphics do not exceed the margin)
\makeatletter
\def\maxwidth{ %
  \ifdim\Gin@nat@width>\linewidth
    \linewidth
  \else
    \Gin@nat@width
  \fi
}
\makeatother

\definecolor{fgcolor}{rgb}{0.345, 0.345, 0.345}
\newcommand{\hlnum}[1]{\textcolor[rgb]{0.686,0.059,0.569}{#1}}%
\newcommand{\hlstr}[1]{\textcolor[rgb]{0.192,0.494,0.8}{#1}}%
\newcommand{\hlcom}[1]{\textcolor[rgb]{0.678,0.584,0.686}{\textit{#1}}}%
\newcommand{\hlopt}[1]{\textcolor[rgb]{0,0,0}{#1}}%
\newcommand{\hlstd}[1]{\textcolor[rgb]{0.345,0.345,0.345}{#1}}%
\newcommand{\hlkwa}[1]{\textcolor[rgb]{0.161,0.373,0.58}{\textbf{#1}}}%
\newcommand{\hlkwb}[1]{\textcolor[rgb]{0.69,0.353,0.396}{#1}}%
\newcommand{\hlkwc}[1]{\textcolor[rgb]{0.333,0.667,0.333}{#1}}%
\newcommand{\hlkwd}[1]{\textcolor[rgb]{0.737,0.353,0.396}{\textbf{#1}}}%
\let\hlipl\hlkwb

\usepackage{framed}
\makeatletter
\newenvironment{kframe}{%
 \def\at@end@of@kframe{}%
 \ifinner\ifhmode%
  \def\at@end@of@kframe{\end{minipage}}%
  \begin{minipage}{\columnwidth}%
 \fi\fi%
 \def\FrameCommand##1{\hskip\@totalleftmargin \hskip-\fboxsep
 \colorbox{shadecolor}{##1}\hskip-\fboxsep
     % There is no \\@totalrightmargin, so:
     \hskip-\linewidth \hskip-\@totalleftmargin \hskip\columnwidth}%
 \MakeFramed {\advance\hsize-\width
   \@totalleftmargin\z@ \linewidth\hsize
   \@setminipage}}%
 {\par\unskip\endMakeFramed%
 \at@end@of@kframe}
\makeatother

\definecolor{shadecolor}{rgb}{.97, .97, .97}
\definecolor{messagecolor}{rgb}{0, 0, 0}
\definecolor{warningcolor}{rgb}{1, 0, 1}
\definecolor{errorcolor}{rgb}{1, 0, 0}
\newenvironment{knitrout}{}{} % an empty environment to be redefined in TeX

\usepackage{alltt}
\usepackage{hyperref}
\usepackage{graphicx}
\usepackage{fullpage}
\usepackage{amsmath}
\IfFileExists{upquote.sty}{\usepackage{upquote}}{}
\begin{document}

\begin{center}
\bf
\LARGE
STATS 406 Fall 2016: Lab 06
\end{center}

\section{R}
\subsection{Vectors}
A vector contains elements of the same type, such as integer, numeric, character, and others. There are various ways to get a subset of a vector. And there are many operations that can be carried out on a vector. Many methods on vectors are the basis for more complicated data structures.
\begin{knitrout}
\definecolor{shadecolor}{rgb}{0.969, 0.969, 0.969}\color{fgcolor}\begin{kframe}
\begin{alltt}
\hlstd{v} \hlkwb{=} \hlkwd{c}\hlstd{(}\hlnum{3}\hlstd{,}\hlnum{1}\hlstd{,}\hlnum{5}\hlstd{,}\hlnum{9}\hlstd{,}\hlnum{7}\hlstd{)}
\hlkwd{print}\hlstd{(v)}
\end{alltt}
\begin{verbatim}
## [1] 3 1 5 9 7
\end{verbatim}
\begin{alltt}
\hlkwd{length}\hlstd{(v)}
\end{alltt}
\begin{verbatim}
## [1] 5
\end{verbatim}
\begin{alltt}
\hlcom{# access elements by []}
\hlstd{v[}\hlnum{1}\hlstd{]}
\end{alltt}
\begin{verbatim}
## [1] 3
\end{verbatim}
\begin{alltt}
\hlstd{v[}\hlnum{1}\hlopt{:}\hlnum{3}\hlstd{]}
\end{alltt}
\begin{verbatim}
## [1] 3 1 5
\end{verbatim}
\begin{alltt}
\hlstd{v[}\hlopt{-}\hlnum{1}\hlstd{]}
\end{alltt}
\begin{verbatim}
## [1] 1 5 9 7
\end{verbatim}
\begin{alltt}
\hlcom{# use logic values to select subset}
\hlstd{v[v}\hlopt{>}\hlnum{4}\hlstd{]}
\end{alltt}
\begin{verbatim}
## [1] 5 9 7
\end{verbatim}
\begin{alltt}
\hlcom{# get the index satisfying certain conditions}
\hlkwd{which}\hlstd{(v}\hlopt{>}\hlnum{4}\hlstd{)}
\end{alltt}
\begin{verbatim}
## [1] 3 4 5
\end{verbatim}
\begin{alltt}
\hlcom{# maximum of a vector}
\hlkwd{max}\hlstd{(v)}
\end{alltt}
\begin{verbatim}
## [1] 9
\end{verbatim}
\begin{alltt}
\hlcom{# index that maximizes the vector}
\hlkwd{which.max}\hlstd{(v)}
\end{alltt}
\begin{verbatim}
## [1] 4
\end{verbatim}
\begin{alltt}
\hlcom{# average of a vector}
\hlkwd{mean}\hlstd{(v)}
\end{alltt}
\begin{verbatim}
## [1] 5
\end{verbatim}
\begin{alltt}
\hlcom{# calculation on logical vectors}
\hlkwd{sum}\hlstd{(v}\hlopt{>}\hlnum{4}\hlstd{)}
\end{alltt}
\begin{verbatim}
## [1] 3
\end{verbatim}
\begin{alltt}
\hlcom{# sort increasingly}
\hlkwd{sort}\hlstd{(v)}
\end{alltt}
\begin{verbatim}
## [1] 1 3 5 7 9
\end{verbatim}
\begin{alltt}
\hlcom{# sort decreasingly}
\hlkwd{sort}\hlstd{(v,} \hlkwc{decreasing} \hlstd{= T)}
\end{alltt}
\begin{verbatim}
## [1] 9 7 5 3 1
\end{verbatim}
\begin{alltt}
\hlcom{# order of elements increasingly}
\hlkwd{order}\hlstd{(v)}
\end{alltt}
\begin{verbatim}
## [1] 2 1 3 5 4
\end{verbatim}
\begin{alltt}
\hlcom{# order of elements decreasingly}
\hlkwd{order}\hlstd{(v,}\hlkwc{decreasing} \hlstd{= T)}
\end{alltt}
\begin{verbatim}
## [1] 4 5 3 1 2
\end{verbatim}
\end{kframe}
\end{knitrout}
  
\subsection{Lists}
A list can store elements that are not necessarily of the same type. Subsetting lists is different from subsetting a vector.
\begin{knitrout}
\definecolor{shadecolor}{rgb}{0.969, 0.969, 0.969}\color{fgcolor}\begin{kframe}
\begin{alltt}
\hlstd{list1} \hlkwb{=} \hlkwd{list}\hlstd{(}\hlkwd{c}\hlstd{(}\hlnum{2}\hlstd{,}\hlnum{5}\hlstd{,}\hlnum{3}\hlstd{),}\hlnum{21.3}\hlstd{,}\hlstr{"hello"}\hlstd{)}
\hlkwd{print}\hlstd{(list1)}
\end{alltt}
\begin{verbatim}
## [[1]]
## [1] 2 5 3
## 
## [[2]]
## [1] 21.3
## 
## [[3]]
## [1] "hello"
\end{verbatim}
\begin{alltt}
\hlcom{# use [[]] to subset lists}
\hlstd{list1[[}\hlnum{1}\hlstd{]]}
\end{alltt}
\begin{verbatim}
## [1] 2 5 3
\end{verbatim}
\begin{alltt}
\hlstd{list2} \hlkwb{=} \hlkwd{list}\hlstd{(}\hlkwd{c}\hlstd{(}\hlnum{1}\hlstd{,}\hlnum{2}\hlstd{,}\hlnum{3}\hlstd{,}\hlnum{4}\hlstd{),}\hlkwd{c}\hlstd{(}\hlnum{5}\hlstd{,}\hlnum{6}\hlstd{,}\hlnum{7}\hlstd{),}\hlkwd{c}\hlstd{(}\hlnum{8}\hlstd{,}\hlnum{9}\hlstd{))}
\hlkwd{print}\hlstd{(list2)}
\end{alltt}
\begin{verbatim}
## [[1]]
## [1] 1 2 3 4
## 
## [[2]]
## [1] 5 6 7
## 
## [[3]]
## [1] 8 9
\end{verbatim}
\begin{alltt}
\hlcom{# use lapply and sapply to run a function on every element of the list}
\hlkwd{lapply}\hlstd{(list2,sum)}
\end{alltt}
\begin{verbatim}
## [[1]]
## [1] 10
## 
## [[2]]
## [1] 18
## 
## [[3]]
## [1] 17
\end{verbatim}
\begin{alltt}
\hlkwd{sapply}\hlstd{(list2,sum)}
\end{alltt}
\begin{verbatim}
## [1] 10 18 17
\end{verbatim}
\end{kframe}
\end{knitrout}

\subsection{Matrices}
A matrix is two-dimensional and stores elements of the same type. There are various ways to subset a matrix and many operations for matrices.
\begin{knitrout}
\definecolor{shadecolor}{rgb}{0.969, 0.969, 0.969}\color{fgcolor}\begin{kframe}
\begin{alltt}
\hlstd{A} \hlkwb{=} \hlkwd{matrix}\hlstd{(}\hlnum{1}\hlopt{:}\hlnum{6}\hlstd{,}\hlkwc{nrow}\hlstd{=}\hlnum{3}\hlstd{,}\hlkwc{ncol}\hlstd{=}\hlnum{2}\hlstd{)}
\hlcom{# In R, the default is by column}
\hlkwd{print}\hlstd{(A)}
\end{alltt}
\begin{verbatim}
##      [,1] [,2]
## [1,]    1    4
## [2,]    2    5
## [3,]    3    6
\end{verbatim}
\begin{alltt}
\hlstd{B} \hlkwb{=} \hlkwd{matrix}\hlstd{(}\hlnum{1}\hlopt{:}\hlnum{6}\hlstd{,}\hlkwc{nrow}\hlstd{=}\hlnum{3}\hlstd{,}\hlkwc{ncol}\hlstd{=}\hlnum{2}\hlstd{,}\hlkwc{byrow}\hlstd{=T)}
\hlkwd{print}\hlstd{(B)}
\end{alltt}
\begin{verbatim}
##      [,1] [,2]
## [1,]    1    2
## [2,]    3    4
## [3,]    5    6
\end{verbatim}
\begin{alltt}
\hlcom{# subset of a matrix}
\hlstd{A[}\hlnum{1}\hlstd{,}\hlnum{2}\hlstd{]}
\end{alltt}
\begin{verbatim}
## [1] 4
\end{verbatim}
\begin{alltt}
\hlstd{A[}\hlnum{2}\hlstd{,]}
\end{alltt}
\begin{verbatim}
## [1] 2 5
\end{verbatim}
\begin{alltt}
\hlstd{A[,}\hlnum{1}\hlstd{]}
\end{alltt}
\begin{verbatim}
## [1] 1 2 3
\end{verbatim}
\begin{alltt}
\hlcom{# elementwise operations}
\hlstd{A}\hlopt{+}\hlstd{B}
\end{alltt}
\begin{verbatim}
##      [,1] [,2]
## [1,]    2    6
## [2,]    5    9
## [3,]    8   12
\end{verbatim}
\begin{alltt}
\hlstd{A}\hlopt{*}\hlstd{B}
\end{alltt}
\begin{verbatim}
##      [,1] [,2]
## [1,]    1    8
## [2,]    6   20
## [3,]   15   36
\end{verbatim}
\begin{alltt}
\hlcom{# matrix multiplication}
\hlstd{A}\hlopt\hlkwd{t}\hlstd{(B)}
\end{alltt}
\begin{verbatim}
##      [,1] [,2] [,3]
## [1,]    9   19   29
## [2,]   12   26   40
## [3,]   15   33   51
\end{verbatim}
\begin{alltt}
\hlcom{# apply a function to every row/column}
\hlkwd{apply}\hlstd{(A,}\hlnum{1}\hlstd{,max)}
\end{alltt}
\begin{verbatim}
## [1] 4 5 6
\end{verbatim}
\begin{alltt}
\hlkwd{apply}\hlstd{(A,}\hlnum{2}\hlstd{,max)}
\end{alltt}
\begin{verbatim}
## [1] 3 6
\end{verbatim}
\end{kframe}
\end{knitrout}

\subsection{Dataframes}
Dataframes are two-dimensional and they allow columns to be of different types. There are some similarities dealing with dataframes comparing to matrices. The following data file ``student-mat.csv'' is from Lab 2.
\begin{knitrout}
\definecolor{shadecolor}{rgb}{0.969, 0.969, 0.969}\color{fgcolor}\begin{kframe}
\begin{alltt}
\hlstd{student} \hlkwb{=} \hlkwd{read.table}\hlstd{(}\hlstr{"student-mat.csv"}\hlstd{,}\hlkwc{header}\hlstd{=T,}\hlkwc{sep}\hlstd{=}\hlstr{";"}\hlstd{)}
\hlcom{# subset of a dataframe}
\hlstd{student[}\hlnum{1}\hlstd{,}\hlnum{3}\hlstd{]}
\hlstd{student}\hlopt{$}\hlstd{age[}\hlnum{1}\hlstd{]}
\hlstd{student[}\hlnum{1}\hlstd{,]}
\hlcom{# select subset satisfying certain conditions}
\hlstd{tmp} \hlkwb{=} \hlstd{student[student}\hlopt{$}\hlstd{age}\hlopt{==}\hlnum{18}\hlstd{,]}
\hlcom{# operations}
\hlkwd{sum}\hlstd{(student}\hlopt{$}\hlstd{age}\hlopt{==}\hlnum{18}\hlstd{)}
\hlkwd{which}\hlstd{(student}\hlopt{$}\hlstd{age}\hlopt{>}\hlnum{18}\hlstd{)}
\hlcom{# average of three grades G1, G2, G3}
\hlstd{avg_G} \hlkwb{=} \hlkwd{apply}\hlstd{(student[,}\hlkwd{c}\hlstd{(}\hlstr{"G1"}\hlstd{,}\hlstr{"G2"}\hlstd{,}\hlstr{"G3"}\hlstd{)],}\hlnum{1}\hlstd{,mean)}
\end{alltt}
\end{kframe}
\end{knitrout}

\subsection{For loops}
For loops can be used to carry out a sequence of operations sharing some common patterns. Sometimes (not always) we can express a for loop in more efficient ways without the loop.

\begin{knitrout}
\definecolor{shadecolor}{rgb}{0.969, 0.969, 0.969}\color{fgcolor}\begin{kframe}
\begin{alltt}
\hlstd{s} \hlkwb{=} \hlnum{0}
\hlkwa{for}\hlstd{(i} \hlkwa{in} \hlnum{1}\hlopt{:}\hlnum{100}\hlstd{)\{}
  \hlstd{s} \hlkwb{=} \hlstd{s} \hlopt{+} \hlstd{i}
\hlstd{\}}
\hlkwd{print}\hlstd{(s)}
\end{alltt}
\begin{verbatim}
## [1] 5050
\end{verbatim}
\begin{alltt}
\hlcom{# another way without using for loop}
\hlkwd{sum}\hlstd{(}\hlnum{1}\hlopt{:}\hlnum{100}\hlstd{)}
\end{alltt}
\begin{verbatim}
## [1] 5050
\end{verbatim}
\end{kframe}
\end{knitrout}

\subsection{While loops}
When the number of iterations is not known, we can use while loops to do a sequence of operations. The code below calculates the largest $K$ such that $\sum_{i=1}^{K}i\leq 200$.
\begin{knitrout}
\definecolor{shadecolor}{rgb}{0.969, 0.969, 0.969}\color{fgcolor}\begin{kframe}
\begin{alltt}
\hlstd{s} \hlkwb{=} \hlnum{0}
\hlstd{i} \hlkwb{=} \hlnum{0}
\hlkwa{while}\hlstd{(s}\hlopt{<=}\hlnum{200}\hlstd{)\{}
  \hlstd{i} \hlkwb{=} \hlstd{i} \hlopt{+} \hlnum{1}
  \hlstd{s} \hlkwb{=} \hlstd{s} \hlopt{+} \hlstd{i}
\hlstd{\}}
\hlstd{K} \hlkwb{=} \hlstd{i} \hlopt{-} \hlnum{1}
\hlkwd{print}\hlstd{(K)}
\end{alltt}
\begin{verbatim}
## [1] 19
\end{verbatim}
\end{kframe}
\end{knitrout}

\subsection{Functions}
We can call and use existing functions in R. Also we can define functions by ourselves. 
\begin{knitrout}
\definecolor{shadecolor}{rgb}{0.969, 0.969, 0.969}\color{fgcolor}\begin{kframe}
\begin{alltt}
\hlstd{f} \hlkwb{<-} \hlkwa{function}\hlstd{(}\hlkwc{n}\hlstd{)\{}
  \hlkwd{return}\hlstd{(}\hlkwd{sum}\hlstd{(}\hlnum{1}\hlopt{:}\hlstd{n))}
\hlstd{\}}
\hlkwd{f}\hlstd{(}\hlnum{100}\hlstd{)}
\end{alltt}
\begin{verbatim}
## [1] 5050
\end{verbatim}
\end{kframe}
\end{knitrout}

\section{SQL}
We can use the language SQL to work with relational databases.

The basic form of ``SELECT'' in SQL is:
\begin{knitrout}
\definecolor{shadecolor}{rgb}{0.969, 0.969, 0.969}\color{fgcolor}\begin{kframe}
\begin{alltt}
\hlstd{sqltext1}\hlkwb{=}\hlstr{"
SELECT ColumnNames
FROM TableName
WHERE Conditions
GROUP BY VariableNames HAVING Conditions
ORDER BY VariableNames;"}
\end{alltt}
\end{kframe}
\end{knitrout}
We can also join two tables to get data, with the following basic form:
\begin{knitrout}
\definecolor{shadecolor}{rgb}{0.969, 0.969, 0.969}\color{fgcolor}\begin{kframe}
\begin{alltt}
\hlstd{sqltext2}\hlkwb{=}\hlstr{"
SELECT T1.ColumnNames, T2.ColumnNames
FROM T1 INNER JOIN T2 ON JoiningConditions;"}
\end{alltt}
\end{kframe}
\end{knitrout}
An example from Lab 4: Using ``baseball.db'', join table \emph{Schools} with table \emph{SchoolsPlayers}. Generate a table showing the number of players from each school in Michigan. Sort by the number of players in descending order.
\begin{knitrout}
\definecolor{shadecolor}{rgb}{0.969, 0.969, 0.969}\color{fgcolor}\begin{kframe}
\begin{alltt}
\hlkwd{library}\hlstd{(}\hlstr{"RSQLite"}\hlstd{)}
\hlstd{driver} \hlkwb{=} \hlkwd{dbDriver}\hlstd{(}\hlstr{"SQLite"}\hlstd{)}
\hlstd{conn} \hlkwb{=} \hlkwd{dbConnect}\hlstd{(driver,} \hlstr{"baseball.db"}\hlstd{)}
\hlstd{sqltext}\hlkwb{=}\hlstr{"SELECT Schools.schoolID, schoolName, COUNT(playerID) AS NumOfPlayers 
                FROM Schools INNER JOIN SchoolsPlayers 
                ON Schools.schoolID = SchoolsPlayers.schoolID 
                WHERE schoolState = 'MI' 
                GROUP BY Schools.schoolID 
                ORDER BY NumOfPlayers DESC;"}

\hlstd{dt} \hlkwb{=} \hlkwd{dbGetQuery}\hlstd{(conn,sqltext)}
\hlkwd{head}\hlstd{(dt)}
\end{alltt}
\begin{verbatim}
##     schoolID                  schoolName NumOfPlayers
## 1   michigan      University of Michigan           72
## 2 michiganst   Michigan State University           37
## 3  wmichigan Western Michigan University           31
## 4  emichigan Eastern Michigan University           14
## 5  cmichigan Central Michigan University           13
## 6    detroit University of Detroit Mercy            8
\end{verbatim}
\end{kframe}
\end{knitrout}

\section{XML}
XML can be used to write documents that store both the data and data descriptions. It has a tree-like format. Note that XML is case-sensitive. An example from Lab 5:
\begin{verbatim}
<?xml version="1.0" encoding="UTF-8"?>
<students>
  <person SocialID="123" SchoolID="92031482">
    <name>Jim Brown</name>
    <gender>Male</gender>
    <major>Mathematics</major>
    <minor> Statistics </minor>
  </person>
  <person SocialID="234" SchoolID="91405720">
    <name>Susan Leicester</name>
    <gender>Female</gender>
    <major>Chemistry</major>
  </person>
</students>
\end{verbatim}
Here ``students'' is the \textbf{root}, which is the first \textbf{node}. Its first child \textbf{node} has \textbf{name} ``person''. This node has two \textbf{attributes} ``SocialID'' and ``SchoolID''.
\end{document}
